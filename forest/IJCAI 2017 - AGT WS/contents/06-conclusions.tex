\section{Conclusions and future research}\label{sec:conclusions}
In this work, we studied the problem of preveting illegal logging in developing countries by selecting and allocating at best a team of heterogeneous resources. Due to the rising threat of this problem in South America and Africa, especially in Madagascar, and the limited budget to face this challenge, the selection, coordination and deployment of the available resources become crucial. Due to the vastness of such environments, the approach proposed in the literature is not able to face this problem when either the pool of resources that can be deployed or the budget increase. To deal with this issue, we first provide some results about the team formation problem when the function of the team is submodular. Then, we exploit such results to design a simple but very effective algorithm based on different heuristics. We test our algorithm, showing that the quality of its performance is high while being significantly fast w.r.t. the exact algorithm, allowing us to tackle bigger and closer--to--reality challenges.

In the future, we will expand our research along two dimensions. On one side, we will try to exploit the structure of the graph to improve both FORTIFY and PATH studying possible adaptations of techniques that exploit properties of the graph, e.g., min-cuts or the degree of the nodes. On the other side, we will study scenarios in which resources can be grouped together according to some of their features, e.g., the type of soil they can move on, such as ground, sea or air, making the model more realistic and exploitable for further applications.

%Currently, we are looking for an exact approach that can scale better than FORTIFY. While it is true that the negative results reported here do not encourage us to take such path, they are just posing limits on the exploitation of the team composition. In fact, there is another path that should be explored, namely the structure of the graph. Exploiting such feature could be crucial, and we are tackling this problem on two sides: on one side, we are studying possible adaptations of techniques that exploit min--cuts or other properties of the graph, e.g., the degree of the nodes; on the other, we are investigating flow problems to see if some methods can be adapted to our case.

%In the future, we will study scenarios in which resources can be grouped together according to some of their features, e.g., the type of soil they can move on, such as ground, sea or air, making the model more realistic and exploitable for further applications.

%Moreover, we will also investigate the complimentary problem the one solved in this paper, namely, suggesting how much budget should be invested to guarantee a given level of security.