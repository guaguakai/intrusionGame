\section{Introduction}\label{sec:introduction}
During the last decades, illegal logging has become one of the major issues both for global forest policy~\cite{tacconi2012illegal} and for several developing countries. According to~\cite{toyne2002timber}, 80\% of logging in the Amazon forests in Brazil violated the government controls, \cite{nellemann2012green} reports that illegal logging contributes to more than 50\% of tropical deforestation in Central Africa, the Amazon Basin and South East Asia and~\cite{wwf15illegal} states that in the Democratic Republic of Congo illegal logging happens at a rate of 65\%, and this value increases up to 80\% and 85\% for Per\'{u} and Myanmar, respectively. This has a strong impact also on the US industry: a study of the American Forest and Paper Association estimated that illegal logging costs the U.S. forest products industry \$1 billion annually in lost export opportunities~\cite{afpa16illegal}. Thus, developing an effective protection of the forests in these areas is a very serious issue for many countries~\cite{allnutt2013mapping,dhital2015issues}. Such protection should be also affordable: these countries have a very limited budget they can spend to solve this problem, making crucial allocating at best the available resources. The present work focus on the deployment of non-homogeneus resources to interdict the traversal of illegal loggers on a network of roads and rivers around the forest areas. We consider non-homogeneus resources since there are different organizations that may be involved in the defensive patrols, e.g., local volunteers, police units, NGO personnel, each differing in their detection skills, both individual and jointly with other patrollers, and costs of deployment. Thus, given some budget, we can build a very large number of teams and, for each of them, we have even more allocation strategies, all with varying effectiveness. Our goal is designing scalable and experimentally reliable algortithms to both find the best team of resources and compute its best allocation in order to maximize the protection of forests against a malicious attacker. To accomplish this task, we exploit game theoretical tools and multiagent systems techniques.

Security Games represent a successful application of non-cooperative game theory in the real world, e.g., \cite{}, and in particular for resource allocation~\cite{korzhyk2010complexity,tambe2011security}. Finding such best allocation is currently one of the most challenging problems in Artificial Intelligence~\cite{jain2012overview}. Customarily, a security game is a 2-player game between a \emph{Defender} \Def~and an \emph{Attacker} \Att, which takes place under a Stackelberg (a.k.a. leader–follower) paradigm~\cite{von2004leadership}, where the Defender (leader) commits to a strategy and the Attacker (follower) first observes such commitment and then best responds to it. As discussed in the seminal work~\cite{conitzer2006computing}, finding a leader-follower equilibrium is computationally tractable in games with one follower and complete information, while it becomes hard in Bayesian games with different types of Attacker. Despite their great effectiveness to solve real problems, the models presented in the literature are quite simple and only few works introduce interaction between the actions played by the Defender and the Attacker, e.g., in~\cite{papadaki2016patrolling} the Attacker observes
the actions of the Defender while in~\cite{bdg2015advpatr,basilico2016multi,debenignis2016} the Defender is supported by an alarm system triggering alarm signals when a target is attacked.

%Starting from this quite simple formulation, literature studied issues like resource scheduling constraints~\cite{DBLP:conf/atal/KiekintveldJTPOT09} and protection of infrastructures~\cite{BlumAAAI14}. On the other side, a very important related research field in multiagent systems is team formation, e.g., in network configuration~\cite{gaston2005agent}, fantasy football~\cite{matthews2012competing} and multi-objective coalition~\cite{cho2013multi}.

We now restrict our attention to Network Security Games (NSG), which constitute the class of games the one studied in this work belongs to, where the Defender must protect some targets placed in an environment usually represented as a graph. ~\cite{jain2011double} proposes a double-oracle approach, i.e., interdiction of adversaries on transportation networks, while~\cite{johnson2012patrol,fang2015security,nguyen2015making} deals with the protection of forests, fish and wildlife. Even thoughthese works deal, once more, with real-life problems, they present two main limitations. First, they only consider the deployment of an already given team of resources, without taking into account the strategic question of the composition of the best team. Second, they only focus on homogeneous resources, i.e., resources with the same cost and skills (see~\cite{jain2013security,okamoto2012solving}). ~\cite{mc2016preventing} proposes an exact algorithm to deal with these limitations. The main problem with this approach is scalability: the approach presented in such work, namely FORTIFY, is exact and, being the problem \textsf{NP}-hard, has an exponential computational cost. In fact, FORTIFY can solve only instances with few resources and a relatively low budget to build the team employed by \Def, being the space of possible teams and their allocations quiet restricted. Thus, the computational issue is still open and very challenging.

%\subsection{Original contributions}
{\bf Original contributions} In order to address scalability and be able to face more challenging and complex real-world scenarios, we provide the following original contributions. First, we introduce the concept of dominance among teams and a new utility function, which allows us to effectively compare different teams. Then, we provide some results about the relations between resources and teams, and among teams themselves. Unfortunately, such results are mostly negative, showing that we cannot exploit the knowledge about the best allocation in the environment for one team to infer information on another one. Such results are not restricted to our domain, but can be applied to any team formation setting where the utility function is submodular (a very common assumption for such problems). Starting from these negative results, we are naturally driven to investigate approximation algorithms, to find good solution in a faster way. Unfortunately, computing an approximate solution to our problem is \textsf{APX}-hard. Nevertheless, we provide some approximation algorithms based on an incremental greedy approach. Finally, we test such algorithms both on synthetic instances and on a real-world instance obtained from joint work with NGOs engaged in forest protection in Madagascar. We show that, despite their simplicity, such algorithms provide a very high utility ratio and are much faster than the exact approach already presented in the literature.